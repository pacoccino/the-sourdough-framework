\begin{quoting}
Dans ce chapitre, vous apprendrez les différents types de pain et leurs
avantages et inconvénients. À la fin de ce chapitre, vous trouverez une recette
très simple de pain plat. C'est probablement le type de pain le plus accessible et le moins
exigeant à faire. Si vous êtes une personne occupée et/ou n'avez pas de
four, c'est peut-être exactement le type de pain que vous devriez envisager.
\end{quoting}

\begin{table}[!htb]
    \begin{center}
        \input{tables/table-overview-bread-types.tex}
        \caption[Différents types de pain]{Un aperçu des différents types de pain
            et leur complexité respective.}%
        \label{tab:bread-types-comparison}
    \end{center}
\end{table}

\section{Pain plat}

Le pain plat est probablement le pain au levain le plus simple à réaliser.
Pour faire un pain plat, aucun four n'est nécessaire ; tout ce dont vous avez besoin est une cuisinière.

\begin{figure}[!htb]
  \includegraphics[width=\textwidth]{sourdough-stove}
  \caption[Einkorn cuit sur feu ouvert]{Un pain plat d'épeautre réalisé directement sur
      le feu. Cela fait partie d'une vidéo où j'essayais de reproduire les recettes de pain au levain
      de nos ancêtres. J'ai appelé la recette "pain de caverne". Certains
      spectateurs ont fait remarquer que probablement pas tous nos ancêtres vivaient dans des grottes.}
\end{figure}

Ce type de pain est super simple à réaliser car vous pouvez sauter
beaucoup de la technique qui est normalement requise. Le pain plat
peut être fait avec toutes sortes de farines. Vous pouvez même utiliser
de la farine sans gluten, comme de la farine de maïs ou de riz, pour réaliser la
pâte. Pour que le pain plat soit un peu plus moelleux, vous
pouvez utiliser un peu de farine de blé. Le gluten en développement
piégera les gaz. Pendant la cuisson, ces gaz vont gonfler la pâte.

Une autre astuce pour améliorer la texture du pain plat est de
faire une pâte très humide. Une grande partie de l'eau s'évaporera
pendant le processus de cuisson et rendra ainsi le pain plus moelleux.

Si votre teneur en eau est très élevée, cela produira une
consistance semblable à celle d'une crêpe.

Consultez la Section~\ref{section:flat-bread-recipe}~``\nameref{section:flat-bread-recipe}''
pour voir une recette complète incluant le processus de fabrication d'un tel pain plat.

\section{Pain de moule}

Le pain de moule est réalisé à l'aide d'un moule à pain spécial
ou d'une boîte à pain. Les bords du moule fournissent un soutien supplémentaire
pour que la pâte monte. Faire un pain à l'aide d'un moule à pain nécessite
un four.

\begin{figure}[!htb]
  \includegraphics[width=\textwidth]{loaf-pan-free-standing.jpg}
  \caption[Pain autonome et pain en moule]{Un pain autonome et un pain de blé
      en moule. Les deux ont reçu une petite incision avant la cuisson
      qui aide à contrôler la manière dont ils s'ouvrent.}%
  \label{fig:free-standing-loaf-pan}
\end{figure}

Après avoir mélangé votre pâte, vous pouvez la placer directement dans le moule à pain.
Cela simplifie tout le processus puisque vous pouvez sauter des étapes telles que
le façonnage de la pâte.

Pour faire un excellent pain de moule avec peu de travail :

\begin{enumerate}
    \item Mélangez les ingrédients de votre pâte (le sans gluten fonctionne aussi)
    \item Mettez dans le moule à pain
    \item Attendez que votre pâte ait à peu près doublé de volume
    \item Cuire dans un four non préchauffé pendant environ 30--50~minutes
\end{enumerate}
Savoir le temps de cuisson exact est parfois un peu difficile car il se peut que l'extérieur de votre pain soit cuit mais l'intérieur soit encore cru. La meilleure manière est d'utiliser un thermomètre et de mesurer la température centrale. À environ  \qty{92}{\degreeCelsius} (\qty{197}{\degF}), votre pâte est prête. Je fais généralement cuire le pain en moule à pain à environ  \qty{200}{\degreeCelsius} (\qty{390}{\degF}), ce qui est un peu moins que mon pain libre que je fais cuire à  \qty{230}{\degreeCelsius} (\qty{445}{\degF}). C'est parce que cela prend un certain temps pour que la pâte cuise correctement à l'intérieur du moule à pain. Les bords ne chauffent pas aussi vite. Puis la partie supérieure de la pâte est correctement cuite, tandis que l'intérieur ne l'est pas encore. Lors de la cuisson, assurez-vous d'utiliser de la vapeur ou placez simplement un autre moule à pain de la même taille au-dessus du vôtre. De cette façon, vous simulez un four hollandais. L'humidité évaporant de la pâte restera à l'intérieur.

Un bon truc pour faire un excellent pain en moule est de faire une pâte très collante. Vous pouvez opter pour une hydratation de \qtyrange{90}{100}{\percent}, ressemblant presque à un levain ordinaire. Tout comme avec le pain plat, l'humidité élevée aide à rendre la mie plus aérée et moelleuse. En même temps, le pain sera un peu plus mâchable. Ce type de pain fait avec du seigle est le style de pain préféré de ma famille. La saveur robuste du seigle associée à la consistance collante fait vraiment un excellent pain à sandwich.

Pour améliorer la structure, vous pouvez également envisager d'utiliser environ \qty{50}{\percent} de farine de blé dans votre mélange. Le réseau de gluten se développera à mesure que votre pâte fermente et permettra de piéger plus de gaz dans la pâte.

Un problème courant que vous rencontrerez lors de la fabrication d'un pain en moule est la pâte qui colle au moule. Utilisez une quantité généreuse d'huile pour graisser votre moule. Un aérosol d'huile végétale antiadhésive peut faire des merveilles. Ne nettoyez pas vos moules à pain avec du savon. Utilisez simplement un torchon pour les nettoyer. Avec chaque cuisson, une meilleure patine se forme, rendant votre moule de plus en plus résistant à l'adhérence.

Ce qui est incroyable avec ce type de pain, c'est qu'il fonctionne avec toutes les farines. Le temps total de travail de la pâte est probablement moins de 5 minutes, ce qui le rend très facile à intégrer dans votre routine quotidienne. De plus, les moules à pain utilisent l'espace dans votre four de manière très efficace. En utilisant des moules, je peux facilement cuire 5 pains en même temps dans mon four domestique. Normalement, j'aurais besoin de plusieurs séances de cuisson pour les pains libres.

\section{Pain libre}

Un pain libre est cuit entièrement sans support de cuisson dans votre four. Pour faire un pain libre, plus d'étapes et d'outils sont nécessaires.

\begin{figure}[!htb]
\begin{center}
  \includegraphics[width=1.0\textwidth]{free-standing-loaf.jpg}
  \caption[Pain au levain libre]{Un pain au levain libre. Notez
      l'incision connue sous le nom de \emph{oreille} et le four qui rebondit clairement
      distinguent ce type de pain du pain plat et du pain en moule.}
\end{center}
\end{figure}

Normalement, vous mélangez votre pâte. Lors de l'utilisation de blé, vous vous assurez de mélanger suffisamment pour développer un réseau de gluten. Vous laissez la pâte atteindre une certaine augmentation de volume pendant la fermentation. Ensuite, vous divisez et préformez la pâte en la forme visuelle que vous préférez. Chaque forme nécessite une technique différente. Parfois, obtenir exactement la bonne forme peut être difficile. Faire une baguette, par exemple, nécessite plus d'étapes. Maîtriser cette technique prend plusieurs tentatives.

Une fois la pâte formée, elle est à nouveau levée pendant une certaine période. Une fois la pâte prête, un outil tranchant comme une lame de rasoir est utilisé pour faire une incision dans la pâte. Cela aide à contrôler comment la pâte s'ouvre pendant le processus de cuisson.

Toutes ces étapes nécessitent de la pratique. Chacune d'elles doit être exécutée parfaitement, sans erreurs. Mais après la cuisson, vous serez récompensé par un beau pain au goût et à la consistance excellents.

Il existe une recette et un tutoriel entièrement dédiés à ce type de pain dans le chapitre~''\nameref{chapter:wheat-sourdough}''.\section{Recette simple de pain plat}%
\label{section:flat-bread-recipe}

Si vous débutez, faire un pain plat est la
façon la plus facile de commencer à faire du bon pain à la maison. Avec juste quelques
étapes, vous pouvez arrêter d'acheter du pain pour toujours. Cela fonctionne avec
n'importe quelle farine, y compris les options sans gluten.

\begin{flowchart}[!htb]
\begin{center}
  \input{figures/fig-process-flat-bread.tex}
  \caption[Processus de pain plat]{Le processus de fabrication d'un pain plat est très
      simple, nécessitant très peu d'effort. Ce type de pain est particulièrement
      pratique pour les boulangers occupés.}%
  \label{fig:flat-bread-process}
\end{center}
\end{flowchart}

C'est ma recette de prédilection que j'utilise pour faire du pain quand
j'ai peu de temps ou quand je suis à l'étranger. Vous pouvez choisir
entre deux options :
%
\begin{enumerate}
    \item Un pain plat similaire à un roti ou un pain naan
    \item des pancakes au levain.
\end{enumerate}

\begin{table}[!htb]
    \begin{center}
        \input{tables/table-flat-bread-pancake-recipe.tex}
        \caption[Recette de pain plat]{Recette de pain plat ou de crêpe pour 1 personne.
            Multipliez les ingrédients pour augmenter la taille des portions.  Référez-vous à la
            Section~\ref{section:bakers-math}
            ``\nameref{section:bakers-math}'' pour apprendre comment comprendre et
            utiliser les pourcentages correctement.}%
            \label{tab:flat-bread-ingredients}
    \end{center}
\end{table}

Pour commencer, préparez votre levain. S'il n'a pas été utilisé depuis très longtemps, envisagez de le nourrir à nouveau. Pour ce faire, prenez simplement \qty{1}{\gram} de votre levain existant et alimentez-le avec \qty{5}{\gram} de farine et \qty{5}{\gram} d'eau. Si vous faites cela le matin, votre levain sera prêt le soir. Plus il fait chaud, plus il sera prêt rapidement. S'il fait très froid là où vous vivez, envisagez d'utiliser de l'eau tiède.

\begin{figure}[htb!]
\begin{center}
  \includegraphics[width=1.0\textwidth]{flat-bread-wheat}
  \caption[Pain plat de blé]{Un pain plat fait uniquement avec de la farine de blé. Le
      La pâte est plus sèche à environ \qty{60}{\percent} d'hydratation. La pâte plus sèche
      est un peu plus difficile à mélanger. Comme le blé contient plus de gluten, la pâte
      gonfle pendant le processus de cuisson.}
\end{center}
\end{figure}

De cette façon, vous devriez avoir environ \qty{11}{\gram} de levain prêt le soir. Vous aurez
la quantité parfaite pour faire une pâte pour une personne. Si vous voulez faire plus
de pain, il suffit de multiplier les quantités indiquées dans
Tableau~\ref{tab:flat-bread-ingredients}.

Puis, le soir, mélangez simplement les ingrédients comme indiqué dans le tableau. Votre pâte
sera prête le matin. Elle est généralement prête après 6--12~heures. Si
vous utilisez plus de levain, il sera prêt plus rapidement. Si vous en utilisez moins, cela prendra
plus longtemps. Essayez de viser un temps de fermentation de 8--12~heures. Si vous utilisez
votre pâte trop tôt, la saveur pourrait ne pas être aussi bonne. Si vous l'utilisez plus tard
votre pâte pourrait être un peu plus aigre. La meilleure option est d'expérimenter
et voir ce que vous aimez personnellement le plus.

Après avoir mélangé les ingrédients ensemble, couvrez le récipient dans lequel
vous avez fait la pâte. Cela empêche la pâte de sécher et fait
assure qu'aucune mouche des fruits n'y ait accès. Un récipient transparent sera utile
lorsque vous débutez. Vous pouvez observer la pâte plus facilement et voir quand
elle est prête.

\begin{figure}[htb!]
\begin{center}
  \includegraphics[width=1.0\textwidth]{ethiopian-woman-checking-bread}
  \caption[Injera éthiopien]{Une femme éthiopienne faisant cuire un \emph{injera}
      fait avec de la farine de teff.  L'image a été fournie par Charliefleurene
      via Wikipedia.}
\end{center}
\end{figure}Si vous avez utilisé l'option de pain plat avec moins d'eau, observez l'augmentation de la taille de votre pâte. La pâte aurait dû augmenter d'au moins \qty{50}{\percent} en taille.
Vérifiez également la présence de bulles sur les côtés de votre récipient.
Lors de l'utilisation de la recette de crêpes, surveillez les bulles à la surface de votre pâte.
Dans les deux cas, utilisez votre nez pour vérifier l'odeur de votre pâte. Selon 
le microbiome de votre levain, votre pâte aura des notes laitières, fruitées, alcoolisées ou vinaigrées, acétiques. Se fier à l'odeur de votre pâte est la meilleure façon de juger si votre pâte est prête ou non. Les timings ne sont pas fiables car ils dépendent de votre levain et de la température. Si votre pâte est prête trop tôt, vous pouvez maintenant la déplacer directement dans le réfrigérateur et la faire cuire à un moment plus opportun. La basse température arrêtera le processus de fermentation\footnote{Il y a quelques exceptions. Dans de rares cas, votre levain peut également fonctionner à des températures plus basses. Vous pouvez avoir cultivé des microbes qui fonctionnent mieux à basse température. Néanmoins, la fermentation est toujours plus lente plus il fait froid. Un réfrigérateur aide vraiment à préserver l'état de votre pâte.} et votre pâte durera plusieurs jours. Plus vous attendrez, plus le pain sera acidulé. Le réfrigérateur est une excellente option si vous voulez emporter la pâte chez des amis. Les gens vont vous adorer pour les pains plats ou les crêpes fraîchement cuits. Si vous osez, vous pouvez aussi goûter un peu de votre pâte crue non cuite. Elle aura probablement un goût relativement acidulé. Je fais cela fréquemment pour mieux évaluer l'état de mes pâtes.


\begin{figure}[htb!]
\begin{center}
  \includegraphics[width=1.0\textwidth]{injera-pancake-texture.jpg}
  \caption[Crêpe de levain de teff]{Une crêpe de levain faite avec de la farine de teff.
      Les poches proviennent de l'eau évaporée et du \ch{CO2} créé par les
      microbes.  L'image a été fournie par Lukasz Nowak via Wikipedia.}
\end{center}
\end{figure}

Si vous vous sentez paresseux ou si vous n'avez pas le temps, vous pouvez également utiliser un vieux levain pour faire directement la pâte sans aucun nourrissage préalable du levain. Votre levain va se régénérer dans votre pâte. Le pain final pourrait être un peu plus acidulé car l'équilibre entre les levures et les bactéries pourrait être perturbé. Dans le tableau~\ref{tab:flat-bread-ingredients}
j'ai recommandé d'utiliser environ \qtyrange{5}{20}{\percent}
de levain en fonction de la farine pour faire la pâte. Si vous suivez
cette approche, utilisez simplement environ \qty{1}{\percent} et faites la pâte directement.
La pâte sera probablement prête 24 heures plus tard, en fonction de la température.

Si vous voulez faire des crêpes sucrées, ajoutez maintenant du sucre et éventuellement des œufs à votre pâte. Une bonne quantité d'œufs est d'environ 1 œuf pour \qty{100}{\gram} de farine.
Remuez un peu votre pâte et elle sera prête à être utilisée. Vous aurez
de délicieuses crêpes sucrées et salées, la combinaison parfaite. En
ajoutant le sucre maintenant, vous vous assurez que les microbes n'ont pas
assez de temps pour le fermenter complètement. Si vous aviez ajouté le sucre
plus tôt, il ne resterait plus de saveur sucrée 12 heures plus tard.

Pour cuire votre pâte, chauffez votre poêle à température moyenne. Ajoutez un peu d'huile à la poêle. Cela aide à la distribution de la chaleur et assure une cuisson uniforme.
Avec une spatule ou une cuillère, placez votre pâte dans la poêle. Si votre pâte
était dans le réfrigérateur, faites-la cuire directement. Il n'est pas nécessaire d'attendre que votre
pâte atteigne la température ambiante. Si vous avez un couvercle,
mettez-le sur votre poêle. Le couvercle aide à cuire votre pâte du dessus.
L'eau qui s'évapore circulera et réchauffera la surface de la pâte. Lorsque
vous faites un pain plat, faites la pâte d'environ \qty{1}{\cm} d'épaisseur. Lorsque vous utilisez l'option
crêpe, optez pour environ \qtyrange{0.1}{0.5}{\cm} selon ce que vous
aimez.\begin{figure}[htb!]
\begin{center}
  \includegraphics[width=1.0\textwidth]{einkorn-crumb.jpg}
  \caption[Einkorn crumb]{La mie d'un pain plat fait avec de l'einkorn comme farine.
      L'einkorn est très pauvre en gluten et ne piège donc pas autant de \ch{CO2} qu'une
      pâte à base de blé. Pour rendre la pâte plus aérée, utilisez plus d'eau ou
      envisagez d'ajouter plus de blé à la préparation de votre pâte.}
\end{center}
\end{figure}

Après 2 à 4 minutes, retournez la crêpe ou le pain plat. Faites-le cuire pendant le même
temps de l'autre côté. Selon ce que vous aimez, vous pouvez attendre un peu
plus longtemps pour permettre au pain de devenir un peu carbonisée. Plus vous
cuisez votre pain, plus l'acidité va s'évaporer. Si votre
pâte est un peu plus sur le côté acide, vous pouvez utiliser cette astuce pour équilibrer
l'acidité. Cela dépend vraiment du goût que vous recherchez.

Lors de la préparation d'un pain plat, je recommande d'envelopper les pains plats cuits
dans un torchon. Ainsi, plus d'humidité évaporée
reste à l'intérieur de votre pain. Cela assure que vos pains plats restent
agréables et moelleux plus longtemps après la cuisson. Une stratégie similaire est
utilisée lors de la préparation des tortillas de maïs.

Vous pouvez conserver en toute sécurité les pains plats ou les crêpes cuits dans votre réfrigérateur
pendant des semaines. Lors de la conservation, assurez-vous de les stocker dans un sac en plastique hermétique afin qu'ils
ne se dessèchent pas.

Gardez un peu de votre pâte crue. Vous pouvez l'utiliser pour faire la prochaine
fournée de pain ou de crêpes pour le lendemain. Si vous voulez cuire quelques jours plus tard, ajoutez
un peu d'eau et de farine et conservez ce mélange dans votre réfrigérateur
aussi longtemps que vous le souhaitez\footnote{Le levain restera bon pendant des mois. Si vous prévoyez de
le laisser plus longtemps, envisagez de sécher un peu de votre levain.}.
