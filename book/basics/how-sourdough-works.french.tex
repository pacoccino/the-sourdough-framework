\begin{quoting}
Dans ce chapitre, nous couvrirons les bases de la fermentation du levain.
Tout d'abord, nous examinerons les réactions enzymatiques qui ont lieu
dans votre farine au moment où vous ajoutez de l'eau, déclenchant le processus de fermentation.
Ensuite, afin de mieux comprendre ce processus, nous 
apprendrons davantage sur les micro-organismes de levure et bactériens impliqués.
\end{quoting}

\begin{figure}[!htb]
  \includegraphics[width=\textwidth]{infographic-enzymes}
  \caption[Interaction des amylases et de la farine]{ Comment les amylases et les protéases
      interagissent avec la farine.}%
  \label{infographic-enzymes}
\end{figure}

\section{Réactions enzymatiques}

Pour comprendre les nombreuses réactions enzymatiques qui ont lieu lorsque la farine
et l'eau sont mélangées, nous devons d'abord comprendre les graines et leur rôle dans
le cycle de vie du blé et des autres céréales.

Les graines sont le principal moyen de reproduction de nombreuses plantes, y compris le blé.
Chaque graine contient l'embryon d'une autre plante, et doit
donc contenir tous les nutriments dont cette nouvelle plante a besoin pour grandir.

Lorsque la graine est sèche, elle est en mode hibernation et peut parfois être
conservée pendant plusieurs années. Cependant, dès qu'elle entre en contact avec l'eau,
elle commence à germer. La graine se transforme en un germe, nécessitant que les
nutriments stockés soient convertis en quelque chose que la plante peut utiliser pendant
qu'elle grandit. Le catalyseur qui rend les réactions associées possibles est l'eau.

La graine contient généralement les premières feuilles prototypiques de la plante,
et elle peut développer des racines en utilisant les nutriments stockés à l'intérieur. Dès que ces feuilles
perce le sol et entre en contact avec la lumière du soleil, elles
commencent à photosynthétiser. Ce processus est le moteur de la plante, et avec l'énergie que produit la photosynthèse, la plante peut continuer à développer plus de racines,
ce qui lui permet d'accéder à des nutriments supplémentaires dans le sol. Ces nutriments supplémentaires permettent à la plante de produire plus de feuilles, augmentant ainsi son activité photosynthétique de sorte qu'elle puisse prospérer dans son nouvel environnement.

Bien sûr, une farine moulue ne peut plus germer. Mais les enzymes qui
déclenchent ce processus sont toujours présentes. C'est pourquoi il est important de ne pas
moudre les grains à une température trop élevée, car cela pourrait endommager certaines
de ces enzymes\footnote{Dans une étude récente~\cite{milling+commercial+home+mill+comparison}, il a été démontré que
moudre de la farine
à la maison avec un petit moulin n'a pas eu d'impact négatif significatif sur la qualité du pain résultant comparé à la farine moulue dans de grands moulins régulés en température.}.

Normalement, l'enveloppe de la graine protège le germe contre les pathogènes. Cependant, comme le
grain est moulu en farine, le contenu de la graine est exposé. C'est idéal
pour nos micro-organismes de levain.

Ni la levure ni les bactéries ne peuvent préparer leur propre nourriture. Cependant, comme
les enzymes sont activées, la nourriture dont ils ont besoin devient disponible, ce qui leur permet
de se nourrir et de se multiplier.

Les deux enzymes principales impliquées dans ce processus sont l'\emph{amylase} et
la \emph{protéase}. Pour des raisons que nous expliquerons bientôt, elles sont de la
plus haute importance pour le boulanger à domicile, et leur rôle dans la fabrication de levain
est une pièce clé du puzzle pour faire du pain meilleur au goût.

\subsection{Amylase}

Parfois, lorsque vous mâchez une pomme de terre ou un morceau de pain pendant une longue période
de temps, vous percevrez un goût sucré sur votre langue. C'est parce que vos
glandes salivaires produisent de l'amylase. L'amylase décompose les molécules d'amidon complexes
en sucres facilement digestibles. Votre corps utilise l'amylase pour commencer le processus de digestion. Le germe fonctionne de manière similaire en utilisant la même enzyme. L'amylase
est utilisée pour créer des sucres à partir de l'amidon pour ensuite produire plus de matière végétale.Normalement,
les microorganismes à la surface du grain ne peuvent pas consommer les molécules de maltose libérées,
qui restent cachées à l'intérieur du germe. Mais lorsque nous broyons la farine, une
frénésie alimentaire a lieu. En général, plus la température est élevée, plus
cette réaction se produit rapidement. C'est pourquoi une longue fermentation est essentielle pour la fabrication d'un bon
pain. Il faut du temps pour que l'amylase décompose la majorité de l'amidon en
sucres simples, qui non seulement sont consommés par la levure mais sont aussi essentiels
à la \emph{réaction de Maillard}, responsable d'une meilleure dorure lors de la
cuisson.

Si vous êtes un brasseur amateur, vous savez qu'il est important de maintenir votre bière à
certaines températures pour permettre aux différentes amylases de convertir les amidons contenus
en sucre~\cite{beer+amylase}. Ce processus est si important qu'il
existe un test fréquemment utilisé pour déterminer si tous les amidons
ont été convertis.
Ce test, appelé le \emph{test d'amidon à l'iode}, consiste à mélanger de l'iode dans
un échantillon de votre infusion et à vérifier la couleur. Si c'est bleu ou noir, vous savez
que vous avez encore des amidons non convertis. Je~me demande si un tel test fonctionnerait également
pour la pâte à pain?

Les boulangers industriels qui ajoutent une levure particulièrement active pour produire du pain en peu de temps
temps font face à un problème similaire. Leur approche consiste à ajouter de la farine maltée à
la pâte, cette farine maltée contient de nombreuses enzymes et accélère ainsi le
processus de fermentation. La prochaine fois que vous serez au supermarché, vérifiez le
l'emballage du pain que vous achetez. Si vous trouvez du \emph{malt} dans la liste des
ingrédients, il y a des chances que cette stratégie ait été utilisée.

Notez qu'il existe en réalité deux catégories de malt. L'une est le \emph{malt
actif enzymatiquement}, qui n'a pas été chauffé à plus de \qty{70}{\degreeCelsius}, où les amylases commencent
à se dégrader. L'autre est le \emph{malt inactif}, qui a été chauffé à des températures plus élevées et n'a donc aucun impact sur votre farine.

\subsection{Protéase}

Tout comme l'amylase décompose les amidons en sucres simples, la protéase décompose
les protéines complexes en protéines plus simples et en acides aminés. Comme le blé
contient du gluten, une protéine essentielle à la structure du pain,
la protéase joue nécessairement un rôle crucial dans la cuisson du levain.

Comme les graines de céréales ont besoin de petits acides aminés pour construire des racines et d'autres
matériaux végétaux, le gluten de ces graines commence à se décomposer dès qu'elles germent, et comme l'ajout d'eau à la farine
actives ces mêmes enzymes,
le même processus se produit dans la pâte à pain.

Si vous avez déjà essayé de faire une pâte à base de blé et que vous l'avez laissée à température ambiante pendant plusieurs jours, vous avez constaté par vous-même que
le réseau de gluten se décompose au point que la pâte ne peut plus tenir ensemble. Une fois
que cela se produit, la pâte se déchire facilement, ne tient pas la structure, et ne
convient plus pour la cuisson du pain.

Cela m'est arrivé une fois lorsque j'ai essayé de faire du levain directement à partir d'un
départ séché. À trois ou quatre jours, la fermentation était si lente que le
réseau de gluten s'est décomposé. La cause profonde de ce problème était la protéase.
En ajoutant de l'eau à votre pâte, vous activez la protéase, et cela se met au travail
en préparant les acides aminés pour le germe.

Voici une autre expérience intéressante que vous pouvez essayer pour mieux visualiser l'
importance de la protéase : Faites une pâte à levée rapide en utilisant une grande quantité
de levure sèche active. En 1 à 2 heures, votre pâte devrait avoir levé et
augmenté de taille. Cuisez-la, puis examinez la structure de la mie. Vous devriez voir
qu'elle est assez dense et loin d'être aussi aérée qu'elle aurait pu l'être. C'est
parce que l'enzyme protéase n'a pas eu assez de temps pour faire son travail.Au début, lors du pétrissage, une pâte devient élastique et se tient très bien. Cependant, à mesure que cette pâte fermente, elle devient plus lâche et extensible~\cite{protease+enzyme+bread}. C'est parce que certaines des liaisons de gluten ont été naturellement décomposées par la protéase à travers un processus connu sous le nom de \emph{protéolyse}. C'est ce qui facilite pour la levure le gonflement de la pâte, et c'est pourquoi un long processus de fermentation est essentiel lorsque vous voulez obtenir une mie aérée et ouverte avec votre pain au levain.

En plus d'utiliser de grands ingrédients, le processus de fermentation lente est l'une des principales raisons pour lesquelles la pizza napolitaine est si savoureuse : parce que la protéase crée une pâte extensible et facile à gonfler, un bord doux et aérée est réalisé.

Comme le processus de fermentation dure généralement plus de 8~heures, une farine ayant une teneur plus élevée en gluten doit être utilisée. Cela donne plus de temps à la pâte pour être décomposée par la protéase sans affecter négativement son élasticité. Si vous deviez utiliser une farine plus faible, vous pourriez finir avec une pâte qui est tellement décomposée qu'elle se déchire lors de l'étirement, rendant impossible, par exemple, de la façonner en tarte à pizza.

Traditionnellement, la pizza a été faite avec du levain, mais de nos jours, elle est faite avec de la levure sèche active. Comme la pâte reste bonne pendant une plus longue période de temps, il est beaucoup plus facile de la manipuler à une échelle commerciale. Si vous deviez utiliser du levain, vous pourriez avoir une fenêtre de trente à quatre-vingt-dix minutes avant que la pâte commence à se détériorer, à la fois à cause de l'action prolongée de la protéase et des sous-produits des bactéries, que nous discuterons plus en détail plus tard dans ce chapitre.

\subsection{Amélioration de l'activité enzymatique}

Comme expliqué précédemment, le malt est une astuce courante pour accélérer l'activité enzymatique. Personnellement, cependant, je préfère éviter le malt et utiliser à la place une astuce que j'ai apprise en faisant du pain complet.

Quand j'ai commencé à faire du pain complet, je n'arrivais jamais à obtenir la croûte, la mie ou la texture que je voulais quel que soit ce que j'essayais. Au lieu de cela, ma pâte avait tendance à trop fermenter assez rapidement. En utilisant une farine blanche avec une teneur en gluten similaire, cependant, mon pain a toujours été très bien.

A l'époque, j'utilisais une autolyse prolongée, qui est juste un mot savant pour mélanger la farine et l'eau à l'avance puis laisser le mélange reposer. La plupart des recettes le prévoient car le processus donne à la pâte une avance enzymatique, et en général c'est une excellente idée. Cependant, comme une alternative tout aussi efficace, vous pourriez simplement réduire la quantité de levure utilisée - dans le cas du levain, ce serait votre levain. Cela permettrait aux mêmes réactions biochimiques de se produire à peu près au même rythme sans vous obliger à mélanger votre pâte plusieurs fois. Mon jeu de blé entier s'est amélioré de façon spectaculaire après que j'ai arrêté d'autolyser mes pâtes.

Maintenant que j'ai eu le temps d'y réfléchir, le résultat que j'ai observé a du sens. Dans la nature, les parties extérieures de la graine entrent en contact avec l'eau en premier, et ce n'est qu'après avoir pénétré cette barrière que l'eau trouve lentement son chemin vers le centre du grain. La graine a besoin de germer d'abord pour surpasser les autres graines à proximité, nécessitant de l'eau pour entrer rapidement. Pourtant, la graine doit aussi se défendre contre les animaux et les bactéries potentiellement dangereuses et les champignons, nécessitant une sorte de barrière pour protéger l'embryon à l'intérieur. Un moyen pour la plante d'atteindre ces deux objectifs serait que la plupart des enzymes existent dans les parties extérieures de la coque. En conséquence, elles sont activées en premier~\cite{enzymatic+activity+whole+wheat}. Par conséquent, en ajoutant simplement un peu de farine complète à votre pâte, vous devriez être en mesure d'améliorer significativement l'activité enzymatique de votre pâte. C'est pourquoi, pour les pâtes de farine blanche, j'ajoute généralement 10 à 20\% de farine de blé entier.

\begin{figure}
  \includegraphics[width=\textwidth]{whole-wheat-crumb}
  \caption{Un pain au levain de blé entier.}%
  \label{whole-wheat-crumb}
\end{figure}En comprenant les deux enzymes clés, l'\emph{amylase} et la \emph{protéase}, vous serez mieux équipé pour faire du pain à votre goût. Préférez-vous une mie plus douce ou plus ferme ? Désirez-vous une croûte plus claire ou plus foncée ? Souhaitez-vous réduire la quantité de gluten dans votre pain final ? Ce sont tous des facteurs que vous pouvez ajuster simplement en modifiant la vitesse de fermentation de votre pâte.

\section{Levure}

Les levures sont des micro-organismes unicellulaires appartenant au royaume des champignons, et des spores vieilles de plusieurs centaines de millions d'années ont été identifiées par des scientifiques. Il existe une grande variété d'espèces - jusqu'à présent, environ \num{1500} ont été identifiées. Contrairement à d'autres membres du royaume des champignons, comme les moisissures, les levures ne créent pas ordinairement un réseau de mycélium.

\begin{figure}[!htb]
\begin{center}
  \includegraphics[width=0.8\textwidth]{saccharomyces-cerevisiae-microscope}
  \caption[Levure de brasseur]{Saccharomyces cerevisiae : Levure de brasseur au microscope.}%
  \label{saccharomyces-cerevisiae-microscope}
\end{center}
\end{figure}

Les levures sont des champignons saprotrophes. Cela signifie qu'elles ne produisent pas leur propre nourriture, mais dépendent plutôt de sources externes qu'elles peuvent décomposer en composés plus facilement métabolisables. Ce que nous appelons aujourd'hui le processus de fermentation, c'est la levure qui décompose les glucides en dioxyde de carbone et en alcool. Ce processus est connu depuis des milliers d'années et a été utilisé depuis l'Antiquité pour la fabrication du pain ainsi que des boissons alcoolisées.

La levure peut se développer et se multiplier à la fois dans des conditions aérobies et anaérobies. Lorsqu'il y a de l'oxygène, elles produisent presque exclusivement du dioxyde de carbone et de l'eau. Lorsqu'il n'y a pas d'oxygène, leur métabolisme change pour produire des composés alcooliques.

Les températures auxquelles la levure se développe varient. Certaines levures, telles que \emph{Leucosporidium frigidum}, se développent mieux à des températures allant de \qty{-2}{\degreeCelsius} à \qty{20}{\degreeCelsius}, tandis que d'autres préfèrent des températures plus élevées. En général, plus l'environnement est chaud, plus le métabolisme de la levure est rapide. La variété de levure que vous cultivez dans votre levain devrait mieux fonctionner dans la plage de températures où le grain a été cultivé et récolté. Ainsi, si vous venez d'un endroit plus frais et que vous cultivez un levain à partir d'une variété de seigle nordique, il y a de fortes chances que votre levure préfère un environnement plus froid.

Par exemple, les brasseurs de bière ont découvert une levure bénéfique vivant dans les grottes froides autour de la ville de Pilsen, en République tchèque. Cette levure est depuis devenue connue pour produire d'excellentes bières à des températures plus basses et des variétés de ces souches sont maintenant utilisées pour brasser des lagers populaires.

Les levures en général sont des organismes très courants. On peut les trouver sur les céréales, les fruits, et de nombreuses autres plantes dans le sol. On peut même les trouver à l'intérieur de votre intestin ! Il se trouve que les types de levure que nous utilisons pour la cuisson sont cultivés sur les feuilles des plantes, bien que très peu soit connu sur l'écologie impliquée.

Les plantes sont protégées par des parois cellulaires épaisses que peu de champignons ou de bactéries peuvent pénétrer. Cependant, certaines espèces produisent des enzymes capables de décomposer ces parois cellulaires pour qu'elles puissent infecter la plante.

Certains champignons et bactéries vivent à l'intérieur des plantes sans leur causer de détresse. On les appelle des \emph{endophytes}. Non seulement ils ne \emph{nuisent pas} à leur hôte, mais ils vivent en fait dans une relation symbiotique. Ils aident les plantes dans lesquelles ils habitent à se protéger d'autres agents pathogènes qui pourraient également venir les infecter par leurs feuilles. En plus de cette protection, ils aident également à résister au stress hydrique et thermique, ainsi qu'à la disponibilité des nutriments. En échange de leurs services à leurs plantes hôtes, ces champignons et bactéries reçoivent du carbone pour l'énergie.Cependant, la relation entre l'endophyte et la plante n'est pas toujours mutuellement bénéfique, et parfois, sous contrainte, ils deviennent des pathogènes envahissants et finalement causent la décomposition de leur hôte.

Il existe d'autres microorganismes qui, contrairement aux endophytes, ne pénètrent pas les parois cellulaires mais vivent plutôt sur la surface de la plante et reçoivent des nutriments de l'eau de pluie, de l'air, ou d'autres animaux. Certains se nourrissent même du miellat produit par les pucerons ou du pollen qui atterrit sur la surface des feuilles. De tels organismes sont appelés des \emph{épiphytes}, et parmi eux se trouvent les types de levures que nous utilisons pour la cuisson.

De manière intéressante, lorsque vous retirez les sources de nourriture externes, un grand nombre de champignons et de bactéries épiphytiques peuvent encore être trouvés sur la surface de la plante, suggérant qu'ils doivent d'une manière ou d'une autre se nourrir directement de la plante. En effet, il y a des recherches indiquant que certaines plantes libèrent intentionnellement des composés tels que des sucres, des acides organiques et aminés, du méthanol et divers sels le long de la surface. Ces nutriments attireraient alors les épiphytes qui vivent sur la surface de la plante.

Les épiphytes sont avantageux pour la survie d'une plante, car ils fournissent une protection renforcée contre les moisissures et autres pathogènes. En effet, il est dans le meilleur intérêt des épiphytes de maintenir leurs plantes hôtes en vie aussi longtemps que possible.

Plus de recherches sont menées chaque jour sur les moyens d'utiliser les levures comme agents de biocontrôle pour protéger les plantes, l'avantage étant que ces bio-agents seraient sans danger pour l'alimentation car les souches pertinentes de levure sont généralement considérées comme inoffensives pour les humains. Les levures se multiplieraient et protègeraient les feuilles, les protégeant essentiellement des autres types de moisissure. Cela pourrait être un changement de jeu potentiel pour les vignobles qui souffrent de l'oïdium.

De tels bio-agents pourraient également être utilisés pour protéger les plantes contre le champignon ergot psychotrope, qui aime pousser dans des environnements plus froids et plus humides et pose un problème significatif pour les agriculteurs de seigle. Les législateurs ont récemment réduit la quantité de contamination par l'ergot autorisée dans la farine de seigle car elle infecte le grain et le rend impropre à la consommation en raison de sa haute toxicité pour le foie. Les levures pourraient aider à atténuer la contamination par l'ergot.

Il y a une autre expérience intéressante réalisée par des scientifiques italiens qui montre combien les levures pourraient être essentielles pour protéger nos cultures. D'abord, ils ont fait de petites incisions dans certains des raisins sur une vigne. Ensuite, ils ont infecté les plaies avec de la moisissure. Certaines incisions n'étaient infectées qu'avec de la moisissure. D'autres étaient également inoculées avec certaines des 150 différentes souches de levures sauvages isolées à partir des feuilles. Ils ont constaté que lorsque la lésion était inoculée avec de la levure, le raisin ne subissait pas de dommages significatifs.

Intriguingly, there was also an experiment performed that showed how brewer's yeast could function as an aggressive pathogen to grapevines. Initially, the yeast lived in symbiosis with the plants, but after the vines sustained heavy damage, the yeast became opportunistic and started to attack, even going so far as to produce hyphae, the mycelium network normally associated with a fungus, so that they could penetrate the tissue of the plants.

\section{Les bactéries}

Les autres antagonistes microbiens dominants dans votre levain sont les bactéries. En fait, elles sont si dominantes qu'elles surpassent les levures dans votre pâte 100 à 1. Alors que la levure fournit le pouvoir de levage, les bactéries créent les saveurs distinctes pour lesquelles le levain a été nommé. Ces dernières sont dues aux sous-produits acides qui résultent de l'alimentation bactérienne. En bonus, ces acides peuvent augmenter significativement la durée de conservation des pains au levain.
Il existe deux types principaux d'acides produits dans le pain au levain : l'acide lactique et l'acide acétique. En termes de saveur, l'acide lactique a des notes clairement laitières, tandis que l'acide acétique a un goût de vinaigre (dont il est d'ailleurs l'ingrédient principal!). Ces sous-produits acides sont produits à la fois par les bactéries lactiques \emph{homofermentatives} et \emph{hétérofermentatives}.

\emph{Homofermentatif} signifie que, pendant la fermentation, la bactérie produit un seul composé : dans ce cas, l'acide lactique. \emph{Hétérofermentatif}, en revanche, signifie que d'autres composés sont également produits : dans ce cas, non seulement de l'acide lactique, mais aussi de l'acide acétique, ainsi que de l'éthanol et même un peu de dioxyde de carbone, deux sous-produits normalement associés à la levure. Une souche assez célèbre de bactéries lactiques, \emph{Fructilactobacillus sanfranciscensis}, tire son nom du tout aussi célèbre pain au levain de style San Francisco. La première culture isolée provenait d'une boulangerie de cette ville, d'où le nom.

La levure et les bactéries sont en concurrence pour la même source de nourriture : le sucre. Certains scientifiques ont rapporté que les bactéries consomment principalement du maltose, tandis que la levure préfère le glucose. D'autres ont rapporté que les bactéries se nourrissent des sous-produits de la levure et vice
versa. Cela est logique, car la nature fait généralement un excellent travail de compostage et de décomposition de la matière biologique~\cite{lactobacillus+sanfrancisco}.

Je n'ai pas encore trouvé une source appropriée qui décrit clairement la symbiose entre
la levure et les bactéries, mais ma compréhension actuelle est qu'elles coexistent et
bénéficient parfois l'une de l'autre, mais pas toujours. La levure, par exemple, tolère l'environnement acide créé par les bactéries environnantes et est donc protégée
contre d'autres agents pathogènes. Cependant, d'autres recherches montrent que les deux
types de microorganismes produisent des composés qui empêchent l'autre de
métaboliser la nourriture --- une observation intéressante, soit dit en passant, car elle pourrait aider à
identifier des antibiotiques ou des fongicides supplémentaires~\cite{mold+lactic+acid+bacteria}.

Dans le passé, j'ai essayé de cultiver des champignons et j'ai observé le mycélium
en train de se défendre contre les bactéries environnantes ; les deux types de
microorganismes produisaient activement des composés pour se combattre l'un l'autre. Et pourtant,
il semblait après un certain temps que le combat avait atteint une impasse, comme si le mycélium avait
entièrement enveloppé la tache bactérienne, l'empêchant de s'étendre davantage. 
J'imagine qu'un scénario similaire pourrait se dérouler dans nos levains, bien que, étant donné que l'environnement du levain a tendance à être plus liquide, ce combat aurait lieu partout dans la pâte et non seulement dans une tache isolée. Des recherches supplémentaires sur ce sujet sont nécessaires pour mieux comprendre
les détails de la relation entre la levure et les bactéries.

Une autre caractéristique intéressante des bactéries du levain à mentionner est leur
capacité à décomposer et à consommer les protéines de votre pâte. Si vous avez déjà fait du pain au levain, il y a de fortes chances que vous ayez expérimenté cela. Vous vous souviendrez de la section \emph{Réactions enzymatiques} que la protéase décompose le
réseau de gluten dans votre pâte, ce qui résulte en un gâchis collant si on le laisse trop longtemps sans cuire. Les bactéries, elles aussi, consomment et décomposent le gluten dans votre
pâte grâce à un processus appelé \emph{protéolyse}.

Cela, pour moi, fut une grande énigme lorsque j'ai commencé à travailler avec du levain.
D'un côté, cela rend la pâte plus collante. De l'autre, cela rend la pâte
plus extensible et plus facile à travailler. Au fur et à mesure que le gluten est réduit, la pâte
devient plus facile à gonfler pour les microorganismes, lui permettant de lever. On peut comparer cela à l'effort nécessaire pour gonfler un pneu épais en caoutchouc
par rapport à un ballon mince et fragile. Le dernier serait facile à gonfler avec
votre bouche, tandis que le premier ne le serait pas.

Sans surprise, la protéolyse est encore accélérée par l'enzyme protéase
précédemment discutée, qui aide à la décomposition du gluten en acides aminés plus petits,
plus facilement métabolisés.Ceci, pour moi, est le processus étonnant de fermentation. Lorsque vous mangez du pain au levain, vous ne consommez pas simplement de la farine et de l'eau mais le résultat final de processus biologiques complexes accomplis par les bactéries et les levures. En raison du composant bactérien ajouté, le pain au levain contient généralement moins de gluten qu'une pâte purement à base de levure~\cite{proteolysis+sourdough+bacteria}. De plus, les bactéries homofermentaires métabolisent l'éthanol produit par la levure et d'autres bactéries lactiques hétérofermentaires. Dans les deux cas, la plupart des composés résultants sont des acides organiques. Chaque ressource naturelle de votre pain au levain est recyclée par les micro-organismes à l'intérieur, qui essaient tous de manger ce qui est disponible aussi longtemps que possible, et à chaque nourrissage, ils deviennent plus aptes à utiliser ces ressources.

Selon le profil de saveur que vous préférez, vous pouvez sélectionner un acide organique ou un autre. La production d'acide acétique nécessite de l'oxygène, et en privant votre levain de celui-ci, vous pouvez augmenter la population de bactéries lactiques homofermentaires. Avec le temps, elles deviendront dominantes et surpasseront les bactéries productrices d'acide acétique~\cite{acetic+acid+oxygen}.

La température de fermentation optimale de vos bactéries lactiques dépend des souches que vous avez cultivées dans votre levain. En général, elles fonctionnent le mieux à la température utilisée pour créer votre levain car vous avez déjà sélectionné des bactéries qui prospèrent dans ces conditions.

Dans une expérience notable, des scientifiques ont examiné les bactéries lactiques présentes sur les feuilles de maïs. Ils ont abaissé la température ambiante de \qtyrange{20}{25}{\degreeCelsius} à environ \qtyrange{5}{10}{\degreeCelsius} et ont ensuite observé des variétés de bactéries qui n'avaient jamais été vues auparavant~\cite{temperature+bacteria+corn}, confirmant qu'il existe en fait une grande variété de souches bactériennes vivant sur les feuilles de la plante.

Incidentalement, vous pourriez faire une expérience similaire en commençant un levain à une température plus basse. En théorie, le microbiome devrait s'adapter, car les micro-organismes qui prospèrent le plus à des températures plus basses commenceront à devenir dominants. Il serait intéressant de voir si cela pourrait influencer activement le goût du pain résultant.

Une dernière note qui mérite d'être mentionnée : certaines sources disent que la fermentation à une température plus basse peut augmenter la production d'acide acétique, tandis que la fermentation à une température plus élevée peut stimuler la production d'acide lactique. Je n'ai pas pu vérifier cela dans mes propres tests. Davantage de recherches sont nécessaires sur le sujet.