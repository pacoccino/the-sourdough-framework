\begin{quoting}
Dans ce chapitre, vous apprendrez à faire un pain au levain de base
en utilisant de la farine non blé. Cela inclut toutes les farines sauf le épeautre.
La différence clé entre la farine de blé et la farine non blé est
la quantité de gluten. Le blé et l'épeautre ont une grande quantité
de gluten. Les farines non blé ne l'ont pas. Dans le cas de la farine de seigle,
des sucres appelés pentosanes empêchent les liaisons de gluten de se former correctement~\cite{rye+pentosans}.
\end{quoting}

\begin{figure}[!htb]
  \includegraphics[width=\textwidth]{final-bread}
  \caption[Pain de seigle au levain]{Un pain de seigle au levain fait en utilisant un moule à pain.
      Le pain de seigle n'est pas marqué. La croûte se fissure généralement pendant
      la cuisson.}%
  \label{fig:non-wheat-final-bread}
\end{figure}

Pour ces farines, y compris le seigle, l'emmer et l'einkorn, aucun développement
de gluten ne doit être fait. Cela signifie qu'il n'y a pas de pétrissage,
pas de sur-fermentation, et pas de problèmes avec la fabrication du pain plat.
Tout le processus
est beaucoup plus facile. Vous mélangez les ingrédients et
attendez pendant une certaine période jusqu'à ce que la pâte ait
atteint le niveau d'acidité que vous aimez. Ensuite, vous
façonnez la pâte ou la versez dans un moule à pain. Après une courte période de repos,
le pain peut être cuit. En raison de l'absence
de développement du gluten, le pain final aura une mie plus dense
comparé au blé.

\begin{flowchart}[!htb]
\begin{center}
  \input{figures/fig-non-wheat-process.tex}
  \caption[Processus pour pain au levain non blé]{Une visualisation du
      processus pour faire du pain au levain non blé. Le processus est beaucoup plus simple
      que de faire du pain au levain de blé. Il n'y a pas de développement de gluten. Les
      ingrédients sont simplement mélangés ensemble.}%
  \label{fig:non-wheat-sourdough}
\end{center}
\end{flowchart}

Ce chapitre se concentrera sur la fabrication du pain de seigle. La farine pourrait
être remplacée par de l'einkorn ou de l'emmer en fonction de vos préférences.

La recette suivante vous permettra de faire 2 pains :
\begin{itemize}
  \item \qty{1000}{\gram} de farine de seigle complète
  \item \qty{800}{\gram} d'eau à température ambiante (\qty{80}{\percent})
  \item \qty{200}{\gram} de levain (\qty{20}{\percent})
  \item \qty{20}{\gram} de sel (\qty{2}{\percent})
\end{itemize}

Le levain peut être dans un état actif ou inactif. Si cela fait
une semaine qu'il est à température ambiante sans nourriture, alors ce sera bon, ou 
s'il vient juste de sortir du réfrigérateur, alors il n'y aura pas de problèmes.
La pâte est très indulgente.

Si vous suivez la pâte suggérée dans la recette, vous faites une pâte de seigle relativement
humide. Elle est si humide qu'elle ne peut être faite qu'à l'aide d'un moule à pain. Si
vous voulez faire un pain de seigle autonome, envisagez de réduire l'hydratation
à environ \qty{60}{\percent}.

\begin{figure}[!htb]
  \includegraphics[width=\textwidth]{ingredients}
  \caption[Pâte non blé]{Pour la pâte non blé, les ingrédients sont mélangés
      ensemble. Il n'est pas nécessaire de développer la force de la pâte. Cela
      simplifie tout le processus de fabrication du pain.}%
  \label{fig:non-wheat-ingredients}
\end{figure}

Mélangez tous les ingrédients avec vos mains. Vous pouvez également
opter pour une spatule pour simplifier les choses. La farine de seigle elle-même est très
collante et désagréable à mélanger à la main. La pâte collera beaucoup
à vos mains. Si vous utilisez un levain dur, il peut être
plus facile de le dissoudre dans l'eau de la pâte. Une fois dissous,
ajoutez les autres ingrédients.

\begin{figure}[!htb]
  \includegraphics[width=\textwidth]{sticky-hands}
  \caption[Pâte de seigle collante]{La farine de seigle contient une molécule de sucre connue sous le nom de pentosane.
      Ces pentosanes empêchent la farine de seigle de former des liaisons de gluten. En conséquence, la pâte ne présente jamais une mie ouverte et est toujours très collante
      lors du mélange à la main.}%
  \label{fig:non-wheat-sticky-hands}
\end{figure}

L'objectif du processus de mélange est d'homogénéiser la pâte. Il n'y a pas besoin de développer de force dans la pâte. Une fois que vous voyez que
votre levain a été correctement incorporé, votre
pâte est prête à commencer la fermentation en masse.Vous pouvez faire fermenter la pâte en masse pendant quelques heures jusqu'à
plusieurs semaines. En prolongeant le temps de fermentation en masse, vous augmentez
l'acidité que le pain final va présenter. Après environ
48 heures, l'acidité n'augmentera plus. C'est parce que
la plupart des nutriments ont été consommés par vos micro-organismes.
Vous pourriez laisser votre pâte reposer plus longtemps, mais cela ne changerait pas beaucoup le
profil de saveur final.

Je recommande d'attendre que la pâte ait environ augmenté de 50 pour cent
en taille. Si vous êtes audacieux, vous pouvez goûter la pâte
pour avoir une idée du profil d'acidité. La pâte aura probablement
un goût très aigre. Cependant, une grande partie de l'acide s'évaporera
pendant le processus de cuisson. Ainsi, le pain final ne sera pas
aussi aigre que la pâte que vous goûtez.

Une fois que vous êtes satisfait du niveau d'acidité, passez à la division
et au façonnage de votre pâte. Il se peut que le façonnage ne soit pas possible si vous optez
pour une pâte plus humide. Si vous avez fait une pâte plus sèche, utilisez autant
de farine que nécessaire pour sécher un peu la pâte et former une boule de pâte.
Il n'est pas question de plier la pâte. Tout ce que vous faites est de la rassembler
autant que nécessaire pour appliquer la forme de votre banneton.
Pour la pâte plus humide, utilisez une spatule et versez autant de pâte que
nécessaire dans votre moule à pain graissé.

\begin{figure}[!htb]
  \includegraphics[width=\textwidth]{crumb}
  \caption[Pain de seigle]{La structure de la mie du pain de seigle. En faisant une pâte plus
  humide, plus d'eau s'évapore pendant la cuisson et donc la
  mie a tendance à être un peu plus ouverte. En général, le
  pain de seigle n'est jamais aussi moelleux que le pain au levain de blé. La croûte
  de ce pain est un peu pâle. La couleur de la croûte peut être contrôlée
  en faisant cuire le pain pendant une période plus longue.}%
  \label{fig:rye-crumb}
\end{figure}

Étalez soigneusement la pâte avec une spatule dans votre moule à pain. Vous
pouvez mouiller la spatule pour faciliter ce processus. Étalez-la
jusqu'à ce que la surface semble lisse et brillante.

Pour le pointage, je recommande d'attendre environ 60 minutes. Une période de pointage prolongée n'a pas de sens à moins que vous ne vouliez augmenter davantage l'acidité de la pâte. La pâte ne deviendra pas plus moelleuse
plus vous laissez pointer. Avec la courte période de pointage, cependant,
la pâte deviendra un peu plus homogène. Ainsi, le pain final a une apparence plus uniforme. La période de pointage permet également à la
pâte de s'étendre pleinement et de remplir les bords du moule à pain. J'aime aussi
mettre la pâte au réfrigérateur pour le pointage. La pâte reste
bonne au réfrigérateur pendant des semaines. Vous pouvez procéder et la cuire à un
moment qui vous convient. 

Une fois que vous êtes satisfait du stade du pointage, procédez et faites cuire votre pâte
comme vous le feriez normalement. Pour plus de détails, veuillez vous référer au 
Chapitre~\ref{chapter:baking}. Un aspect difficile
lors de l'utilisation d'un moule à pain est de s'assurer que le centre de votre pâte est correctement cuit. Pour cette raison, il est préférable d'utiliser un thermomètre
et de mesurer la température interne. Le pain est
prêt lorsque la température interne atteint  92 degrés Celsius (197 degrés F). Je recommande
de retirer le pain du moule à pain une fois qu'il atteint la température souhaitée. Ensuite, vous pouvez continuer à cuire le pain sans le moule et
la vapeur. De cette façon, vous obtiendrez une belle croûte tout autour de votre
pain. Vous pouvez cuire aussi longtemps que vous le souhaitez jusqu'à ce que vous ayez obtenu
la couleur de croûte de votre choix. Plus elle est foncée, plus la croûte est croustillante
et offre plus de saveur. Si vous pensez que votre
pâte a pu être trop acide, vous pouvez prolonger le temps de cuisson.
Plus vous cuisez longtemps, plus l'acidité s'évaporera.

C'est l'un de mes pains préférés à cuire que je mange presque tous les
jours. L'effort nécessaire pour faire du pain comme
celui-ci est beaucoup plus faible comparé à une pâte à base de blé. Dans certains
cas, je prolonge la recette et j'ajoute du levain supplémentaire
à la pâte. Vous pouvez ajouter autant de levain que vous le souhaitez. Le pain résultant
a un profil de saveur très complexe mais délicieux.