\chapter{L'histoire du levain}%
\label{ch:history}
\begin{quoting}
    Nous commencerons ce livre en parlant brièvement de la longue histoire du
    pain au levain depuis l'antiquité, et comment les gens ont utilisé des procédés similaires pour
    d'autres aliments comme la bière. La découverte de la levure et comment, conjointement avec
    le développement de la machine, elle a révolutionné la fabrication du pain. Plus récemment
    des communautés se sont formées autour du levain et de la cuisson à la maison, essayant de réapprendre
    les leçons du passé.
\end{quoting}

Le levain a été préparé depuis l'Antiquité. Les origines exactes du pain fermenté
sont cependant inconnues. L'un des pains au levain les plus anciens conservés
a été retrouvé en Suisse. 
Cependant, sur la base de recherches récentes, certains scientifiques supposent que le pain au levain
avait déjà été fabriqué en 12000 av. J.-C. en Jordanie ancienne~\cite{jordan+bread}.

\begin{figure}[ht]
  \includegraphics[width=\textwidth]{einkorn-crumb}
  \caption[Ancien pain plat à l'einkorn]{Un ancien pain plat à l'einkorn. Notez la
      structure dense de la mie.}%
  \label{einkorn-crumb}
\end{figure}

Une autre histoire populaire raconte qu'en Egypte, une femme préparait
une pâte à pain près du fleuve Nil. La femme a oublié la
pâte et à son retour quelques jours plus tard, elle a remarqué que la pâte avait
augmenté de volume et sentait bizarre. Elle a donc décidé de cuire
la pâte et a été récompensée par un pain
beaucoup plus léger, plus doux, au goût plus agréable. A partir de ce jour,
elle a continué à faire du pain de cette manière.

Les gens de l'époque ignoraient que de minuscules microorganismes
étaient la raison pour laquelle le pain était meilleur. On ne sait pas quand
ils ont commencé à utiliser un peu de la pâte de la veille
pour la prochaine fournée. Mais en le faisant, la fabrication
du pain au levain est née : La levure sauvage dans la farine et dans l'air
plus les bactéries commencent à décomposer le mélange farine-eau, aussi
connu sous le nom de pâte. La levure rend la pâte moelleuse et
les bactéries créent principalement de l'acidité. Les différents
microorganismes travaillent en symbiose. Les humains
appréciaient la structure aérée améliorée et la légère acidité
de la pâte. De plus, la durée de conservation de ce pain
était prolongée en raison de l'acidité accrue.

Rapidement, des processus similaires ont été découverts pour la fabrication de la bière
ou du vin. Un petit lot de la production précédente
servait pour la prochaine production. De cette façon, les humains ont créé
les levures modernes pour le pain, le vin et la bière. Ce n'est qu'en 1680
que le scientifique Anton van Leeuwenhoek a étudié pour la première fois les microorganismes de la levure
sous un microscope. Au fil du temps, à chaque fournée, les levures et les bactéries
devenaient de plus en plus efficaces pour consommer ce qu'on leur donnait.
En nourrissant votre starter de levain, vous sélectionnez
des microorganismes qui sont bons pour manger votre farine. Avec
chaque itération, votre levain sait comment mieux fermenter la farine
qui lui est donnée. C'est aussi la raison pour laquelle les starters de levain plus âgés ont parfois tendance à faire lever la pâte plus rapidement~\cite{review+of+sourdough+starters}. C'est fou si vous
y pensez. Les gens ont utilisé ce processus sans
savoir ce qui se passait réellement pendant des milliers d'années! Le
levain en lui-même est une relation symbiotique. Mais le levain
s'est également adapté aux humains et a formé une relation symbiotique avec nous.
En échange de nourriture et d'eau, nous sommes récompensés par un délicieux pain. En échange,
nous abritons et protégeons le levain. Les spores du starter
se propagent par contamination aérienne ou par des insectes comme les mouches des fruits.
Ainsi, le starter de levain peut propager ses spores
encore plus loin à travers le monde.

Les brasseurs ont commencé à expérimenter en utilisant les restes boueux
de la fermentation de la bière pour commencer à faire des pâtes. Ils ont remarqué
que les pâtes à pain résultantes devenaient moelleuses et manquaient d'acidité par rapport au processus de levain,
dans le produit final.
Un exemple populaire est montré dans un rapport de 1875. Eben Norton Horsford
a écrit à propos du fameux \emph{Kaiser Semmeln} (petits pains de l'empereur).
Il s'agit essentiellement de petits pains faits avec de la levure de bière au lieu
de l'agent levant de levain. Comme le processus est plus coûteux,
des petits pains comme ceux-ci ont finalement été consommés par les nobles
à Vienne~\cite{vienne+petitspains}.

\begin{figure}[ht]
  \includegraphics[width=\textwidth]{sourdough-stove}
  \caption{Un pain fait sur le poêle sans four.}%
  \label{sourdough-stove}
\end{figure}

Ce n'est qu'en 1857 que le microbiologiste français Louis Pasteur a découvert
le processus de fermentation alcoolique. Il prouverait que
les micro-organismes de la levure sont la raison de la fermentation alcoolique
et non d'autres catalyseurs chimiques. Ce qui commencerait alors,
c'est ce que je décris comme les 150 années perdues de la fabrication du pain. En 1879,
les premières machines et centrifugeuses ont été développées pour centrifuger
de la levure pure. Cette levure serait extraite de lots de levain.
La levure pure s'est révélée excellente et turbocompressée
pour faire lever les pâtes à pain. Ce qui aurait pris 10 heures
pour faire lever une pâte à pain peut maintenant être fait en 1 heure.
Le processus est devenu beaucoup plus efficace. Pendant la Seconde Guerre mondiale,
la première levure sèche emballée a été développée. Cela permettrait finalement
aux boulangeries et aux boulangers à domicile de faire du pain beaucoup plus rapidement.
Grâce à la levure pure, la construction de machines à pain était
possible. À condition de maintenir la même température,
votre levure ferait toujours fermenter de la même manière.

Comme les temps de fermentation
se sont accélérés, le goût du pain final s'est détérioré.
Le processus de germination induit par certaines enzymes est essentiel
pour développer une texture plus moelleuse et une croûte meilleure. Cela
ne peut pas être accéléré indéfiniment. Bientôt, les boulangeries commenceraient
à introduire des enzymes supplémentaires pour obtenir des propriétés similaires
au pain au levain dans les pâtes à base de levure. Le levain a presque complètement
disparu de la surface de la Terre. Seuls une poignée
de vrais nerds continueraient à faire du pain avec du levain.
Soudain, les gens ont commencé à parler plus souvent de la maladie cœliaque et du rôle du gluten. La maladie n'est pas nouvelle ; elle a été décrite pour la première fois en 250 après J.-C. Les gens notaient comment le pain moderne contient beaucoup plus de gluten comparé au pain ancien. Le pain dans l'antiquité était probablement beaucoup plus plat. Les céréales ont été de plus en plus cultivées pour contenir une plus grande quantité de gluten. Le gluten est une protéine qui donne à notre pain moderne sa structure typique, moelleuse et aérée. Les protéines du gluten se lient ensemble lorsqu'elles sont activées par l'eau. Au cours de la fermentation, le \ch{CO2} est piégé dans cette matrice de protéines. Les petites cavités créées se dilatent pendant le processus de cuisson. Comme la pâte se gélifie en chauffant, la structure est renforcée. Cela rend le pain doux et moelleux en bouche. Tout comme boire du lait de vache cru, votre système immunitaire peut réagir aux protéines consommées. Il y a l'intolérance au gluten et la maladie cœliaque. Quand les gens disent qu'ils ne tolèrent pas bien le gluten, c'est généralement une intolérance au gluten qu'ils décrivent. Certaines personnes décrivent des problèmes similaires lorsqu'elles consomment trop de lactose. Si vous mangez un fromage longuement fermenté, la plupart du lactose a été fermenté par les petits micro-organismes. Les gens ont noté que le pain au levain peut généralement être mieux toléré comparé au pain de fabrication rapide. La raison en est que les enzymes prennent du temps pour travailler la pâte. Le gluten est une protéine de stockage de la farine. Une fois que la germination est activée par l'ajout d'eau, l'enzyme protéase commence à convertir le gluten en petits acides aminés nécessaires à la germination. Au fil du temps, vous perdez effectivement du gluten car il est naturellement décomposé. De plus, traditionnellement, les bactéries lactiques commencent à décomposer le mélange farine-eau. Presque tout est recyclé dans la nature. Elles consomment les protéines présentes dans la pâte. Le pain moderne est plus rapide et ne contient plus de bactéries lactiques. Ces deux facteurs signifient que vous consommez des produits avec une bien plus grande valeur en gluten comparée à l'époque ancienne où la fermentation naturelle était utilisée.

Pendant la ruée vers l'or en Californie, les boulangers français ont apporté la culture du levain en Amérique du Nord. Un pain populaire est devenu le levain de San Francisco. Il est caractérisé par son goût unique (auparavant commun à tous les pains). Cependant, il est resté plutôt un aliment de niche. Ce qui a véritablement accéléré le retour du levain a été la pandémie de COVID-19 en 2020. La farine et la levure sont devenues rares dans les supermarchés. Alors que la farine est revenue, la levure n'a pu être trouvée. Les gens ont commencé à chercher des alternatives et ont redécouvert l'ancienne méthode de fabrication du pain au levain. Beaucoup ont vite réalisé que la fabrication du pain au levain est plus complexe que celle du pain moderne à base de levure. Vous devez entretenir un levain et le maintenir en bonne forme pour fermenter correctement votre pâte. De plus, contrairement à une pâte à base de levure, vous ne pouvez pas simplement faire tomber la pâte et laisser la fermentation continuer. Vous pouvez faire surfermenter votre pâte, ce qui résulte en un pétrin collant. Cette complexité a mené de nombreux boulangers à chercher de l'aide et de nombreuses communautés prospères se sont formées autour du sujet du pain fait maison.

Lors d'une interview de Karl de Smedt (propriétaire de la Bibliothèque du levain) il a dit quelque chose qui a changé ma façon de penser le pain : "L'avenir du pain moderne est dans le passé".