\begin{quoting}
Dans ce chapitre, nous examinerons de plus près différents types de farine
et leur catégorisation respective. Nous verrons aussi comment distinguer couramment
les différentes farines du même type. Ainsi, vous pourrez acheter plus confiant
la farine dont vous avez besoin.
\end{quoting}

Le type de farine le plus basique est une farine de grain entier. Dans ce cas, toute la graine a
été broyée en petits morceaux. Parfois, selon ce que vous voulez cuire,
le goût corsé du son peut ne pas être désiré. Dans ce cas, vous pouvez utiliser
des farines plus blanches. Avec des tamis, les moulins enlèvent une grande partie de la coque de la graine.
La graine contient déjà un germe préconstruit de la plante qui attend d'être
activé. La farine la plus blanche que vous pouvez obtenir est principalement juste la partie amidon de la graine.
Selon les couches encore présentes, des noms sont utilisés pour décrire le
type de farine.

\begin{table}[!htb]
    \begin{center}
        \begin{tabular}{@{}llrrr@{}}
\toprule
\textbf{USA}  & \textbf{UK}  & {\textbf{Germany}} & {\textbf{France}} & {\textbf{Italy}} \\ \midrule
Cake         & Soft flour  &  T405            &  T45        & 00 \\ 
All purpose  & Plain flour &  T550            &  T55        & 0 \\ 
Bread flour  & Bread flour &  T405 or T550    &  T45 or T55 & 00 or 0 \\ 
             &             &  T812            &  T80        & 1 \\ 
             &             &  T1050           &  T110       & 2 \\ 
Whole        & Whole       &  Vollkorn        &  T150       & Integrale \\ \bottomrule
\end{tabular}

        \caption[Étiquetage de la farine de blé]{Une comparaison de comment différents types
            de farine de blé sont étiquetés dans différents pays.}%
        \label{tab:flour-types-comparison}
    \end{center}
\end{table}

En Allemagne, la teneur en cendres est utilisée pour décrire les farines. Le laboratoire va brûler
\qty{100}{\gram} de farine au four. Ensuite, les cendres restantes sont extraites
et mesurées. En fonction de la quantité, la farine est catégorisée. Si la farine
est de type 405, alors \qty{405}{\mg} de cendres ont subsisté après la combustion de la
farine. Plus la farine a de parties de cosse, plus de minéraux restent. Donc plus
le nombre est élevé, plus la farine est proche de la farine complète. Les chiffres sont
légèrement différents pour chaque type de céréale. En général, cependant, plus la
valeur est élevée, plus le goût sera corsé.

\begin{figure}[htb!]
  \includegraphics[width=\textwidth]{wheat-kernel-overview}
  \caption[Contenu d'un grain de blé]{Un aperçu d'un grain de blé avec
      son contenu~\cite{wheat+kernel}.}%
  \label{fig:wheat-kernel-overview}
\end{figure}

Si vous comparez différents types de céréales, il y a des céréales à gluten élevé, faible gluten
et sans gluten. Le gluten est ce qui permet au pain d'avoir sa consistance moelleuse.
Sans gluten, les produits de boulangerie n'auraient pas les mêmes propriétés. Gérer
le gluten rend tout le processus de fabrication du pain plus complexe car plus d'étapes sont impliquées.
Une pâte sans gluten n'a pas besoin d'être pétrie. Le pétrissage crée
les liens de gluten. Plus vous pétrissez, plus ils deviennent forts. Avec les farines
à faible teneur en gluten et sans gluten, vous avez seulement à mélanger les ingrédients ensemble, en veillant à bien tout homogénéiser. Pendant la fermentation
le gluten se dégrade car les microorganismes le métabolisent. Quand trop de gluten
a été converti, votre pâte n'aura plus la structure de type blé précédemment
décrite. Pour les farines sans/à faible teneur en gluten, votre principal souci est de gérer l'acidité. Vous ne voulez pas
que le pain final soit trop acide. Vous n'avez pas à vous soucier de la dégradation du gluten,
ce qui vous évite un gros mal de tête.

\begin{table}[!htb]
    \begin{center}
        \begin{tabular}{@{}lcccc@{}}
\toprule
\textbf{Grain type}        & \textbf{Homogenize} & \textbf{Knead} & \textbf{Stretch \& Fold} & \textbf{Shape} \\ \midrule
Spelt, Wheat (\textgreater{}~70\%) & Yes & Yes & Yes & Yes \\
Rye, Emmer, Einkorn, Rice, Corn    & Yes & No  & No  & No  \\ \bottomrule
\end{tabular}

        \caption[Différents types de céréales]{Un aperçu des différents types de céréales
          et les étapes impliquées dans le processus de fabrication du pain respectif.}
    \end{center}
\end{table}

Comme le gluten a un rôle spécial, le reste de ce chapitre est consacré à examiner de plus près les différentes farines à gluten et comment les distinguer. L'épeautre
contient également des quantités significatives de gluten, donc les mêmes caractéristiques sont valables.Plusieurs recettes nécessitent de la farine de blé pour pain. La farine à pain peut faire référence à différents types de farine. Elle pourrait être une T405 ou une T550 en Allemagne. Ceci est très souvent classifié de manière incorrecte. Les termes \emph{forte} ou \emph{à pain} se réfèrent dans ce cas aux propriétés de la farine. Une farine à pain est considérée comme ayant une quantité plus élevée de protéines et donc de gluten. Cette farine est excellente lorsque vous voulez faire un pain au levain car votre pâte permet une période de levée plus longue. Comme décrit précédemment, le gluten est consommé par vos micro-organismes. Plus vous avez de gluten, plus votre pâte conserve son intégrité. Si vous vouliez faire un gâteau, vous voudriez peut-être utiliser une farine avec moins de gluten. Les propriétés liantes du gluten pourraient ne pas être souhaitables car le gâteau final pourrait avoir une texture mâchable.

En conclusion, toutes les farines T405, T45 ou T00 ne sont pas les mêmes. En fonction des propriétés de la plante dont elles sont issues, les farines auront des propriétés différentes. Pour cette raison, certains pays comme l'Allemagne ont introduit des échelles supplémentaires pour évaluer la qualité du blé. La catégorie \textbf{A} fait référence au blé de bonne qualité qui peut être mélangé avec des qualités inférieures pour améliorer la farine. La catégorie \textbf{B} fait référence au blé moyen qui peut être utilisé pour créer différents produits de boulangerie. La catégorie \textbf{C} est utilisée pour le blé qui a de mauvaises qualités de cuisson. Cela pourrait arriver, par exemple, si le blé a déjà commencé à germer et a donc perdu certaines de ses propriétés intéressantes pour la cuisson. Ce type de blé est généralement utilisé dans l'alimentation animale ou comme biomasse fermentable pour les générateurs. La catégorie \textbf{E} fait référence au blé \emph{Élite}. C'est la meilleure qualité de blé. Ce type de blé ne peut être récolté que lorsque le blé a poussé dans des conditions optimales. Vous pouvez le comparer à un vignoble qui n'utilise que les meilleurs raisins pour faire un vin de réserve. Malheureusement, cela n'est normalement jamais imprimé sur l'emballage de la farine que vous achetez. Vous pouvez chercher la valeur en protéines comme un indicateur possible. Cependant, les grandes meuneries mélangent les farines pour maintenir la qualité tout au long des années. La farine mélangée n'est pas non plus indiquée sur l'emballage. Il se peut que les boulangeries extraient du gluten de certaines farines et le mélangent afin de créer de meilleures farines à cuire.

En Italie, la soi-disant \textbf{valeur-W} a été introduite pour mieux montrer comment la farine se comportera. Une pâte est fabriquée, puis la résistance de cette pâte au pétrissage est mesurée. Plus une farine a de gluten, plus la pâte est élastique, et plus elle résistera au pétrissage. Une farine à W plus élevée aura une teneur en gluten plus élevée et permettra une période de fermentation plus longue. Mais en même temps, il est aussi plus difficile pour les microbes de gonfler la pâte car il y a plus de matériel de ballon. Pour faire un excellent produit fermenté à partir d'une farine à W élevée, vous devrez avoir une longue période de fermentation. Cette longue période de fermentation signifie aussi que vos microbes enrichiront votre pâte avec plus de saveur.

\begin{table}[!htb]
    \begin{center}
        \begin{tabular}{@{}rcll@{}}
\toprule
\textbf{W-Value} & \textbf{Hydration (\%)} & \textbf{Uses} & \textbf{Fermentation time} \\ \midrule
0--150          & 50     & Cookies             & Very short    \\ 
150--250        & 50--60 & Cakes, Bread, Pizza & Short--Medium \\ 
250--350        & 60--70 & Bread, Pizza        & Long          \\ 
350+            & 70--90 & Bread, Pizza        & Very long     \\ \bottomrule
\end{tabular}

        \caption[Durée de fermentation versus valeur-W]{Un aperçu des différents
            niveaux de valeurs W et des hydrats et durées de fermentation respectifs.}%
        \label{tab:w-value}
    \end{center}
\end{table}

En général, lorsque l'on vise à cuire du pain au levain sans moule, il faut viser une teneur plus élevée en protéines. Si la valeur du gluten est relativement faible, votre pain s'effondrera plus rapidement. Il est toujours possible de faire du pain, mais il pourrait être plus facile d'utiliser des outils tels qu'un moule à pain ou de faire du pain qualifié ou du pain plat.Une caractéristique supplémentaire, rarement considérée, d'une bonne farine est le niveau de dommages aux molécules d'amidon. C'est un problème courant lorsque vous essayez de moudre vos propres farines de blé à la maison. Les chances sont que votre moulin à domicile n'est pas capable d'atteindre les mêmes résultats qu'un moulin plus grand. L'endommagement des amidons est essentiel pour améliorer les propriétés de la pâte. Vous aurez une meilleure gélatinisation et absorption d'eau avec de l'amidon correctement endommagé~\cite{starch+damage+flour}. Plus d'amidon est endommagé, plus la surface augmente. Cela améliore la façon dont l'eau interagit avec la farine. Cela fournit également une plus grande surface que vos microbes peuvent utiliser pour attaquer les molécules et commencer le processus de fermentation.

Je n'ai pas encore trouvé un bon moyen de moudre ma propre farine à la maison. Même après avoir essayé de moudre la farine 10 fois avec de courtes pauses, je n'ai pas pu obtenir les mêmes propriétés qu'avec de la farine moulue commercialement. Les pâtes que je ferais  se sentait bien, peut-être un peu grossière. Cependant, pendant la cuisson, les pâtes commençaient à dégazer rapidement et se transformaient en pains très plats. J'ai eu un grand succès cependant en utilisant de la farine moulue à la maison avec un moule à pain ou comme pain de poêle. Si vous avez trouvé de bonnes façons de travailler avec de la farine moulue à la maison, s'il vous plaît contactez-moi. Le potentiel d'utilisation des farines moulues à la maison est énorme. Cela permettrait même aux communautés éloignées de cultiver leur propre blé et de pouvoir produire un pain fraîchement cuit incroyable.