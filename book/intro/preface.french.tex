\chapter{Préface}%
\label{ch:Preface}

S'il y a un aliment pour lequel l'Allemagne est connue, c'est probablement le pain.
Il existe des milliers de variétés en Allemagne,
et sa fabrication fait partie intégrante de notre culture.

Mon périple du pain a commencé durant mon enfance. Ma mère, étant parent
de 3 enfants, profitait toujours des samedis pour cuire une délicieuse miche pour la famille.
C'était un pain de mie blanc et moelleux, et elle le faisait en une à deux heures en utilisant de la levure achetée en magasin.
Étant un peu plus expérimenté maintenant, je réalise qu'il est
idéal d'attendre un peu avant de trancher votre pain, mais à l'époque,
nous, les enfants, ne pouvions pas attendre. Maman nous coupait quelques tranches directement sorties du four, et nous commencions immédiatement
à verser du beurre ou de la confiture sur chaque tranche. En quelques minutes,
un kilogramme de farine était consommé. Le pain est devenu une partie intégrante de mon
alimentation hebdomadaire.

J'ai eu la chance que mes parents puissent se permettre un voyage de ski annuel à
Alto Adige dans le nord de l'Italie. Dans la petite ville appelée Valdaora, nous
essayions chaque année de nouveaux restaurants, pour finir toujours dans notre pizzeria
préférée. Les pizzas y étaient incroyables. La pâte
était si savoureuse que nous commandions juste le pain avec un
peu d'huile d'olive et de sel.

Bien sûr, ma question serait toujours: "Maman, pouvons-nous faire cela à la maison aussi, s'il te plaît ?"
Donc, au fil des années, nous sommes devenus amis avec les propriétaires et nous recevions
de plus en plus d'indices sur la façon de faire la pâte à pizza parfaite. Il
n'y a pas d'ingrédients secrets à l'intérieur. Ce n'est que de la farine, de l'eau, du sel et un peu de levure.
Comment une combinaison aussi simple d'ingrédients peut-elle créer une pâte à pizza
incroyablement délicieuse ? Mes parents, étant des créatures d'habitude, revenaient chaque année avec nous,
et chaque année, mon intérêt grandissait. À la maison, maman et moi avons tenté de reproduire
la recette. Nous avons essayé de cuire sur une pierre et sur un acier. Nous avons essayé d'ajouter de l'huile à la pâte et des herbes
à la sauce à pizza. Nous sommes tombés dans un cycle sans fin d'expériences. Cependant, nous n'avons jamais réussi
à nous rapprocher de l'expérience que nous avions pendant les vacances.

Quelques années ont passé, et j'ai finalement commencé mes études dans la petite ville allemande de Göttingen.
Pour la première fois, je devais faire mes propres courses de pain. Je n'avais jamais
envisagé de commencer à le faire moi-même. J'achetais simplement
un bon pain en faisant mes courses au supermarché. Ma variété préférée
était un \emph{Schwarzbrot : Korn an Korn}. C'est un pain de seigle très foncé et nourrissant
avec des baies et des graines de tournesol ajoutées.

Étant un peu naïf, je n'avais jamais examiné l'emballage de ce que j'achetais
auparavant. Un jour, cela a changé. J'ai regardé l'étiquette et j'ai été choqué. Le
pain apparemment sain était composé de tant d'autres choses à part de la farine et de l'eau.
La couleur noire ne venait pas de la farine, mais du sucre caramélisé.
L'emballage indiquait qu'il s'agissait d'un pain au levain, mais alors pourquoi y avait-il de la levure supplémentaire?
J'ai pensé que si c'était vraiment du levain, il ne devrait pas nécessiter de levure supplémentaire, et j'ai vite
réalisé que quelque chose clochait avec le pain que j'achetais.
J'ai procédé à la vérification des autres pains du supermarché, pour découvrir qu'eux aussi,
contenaient des ingrédients que je n'avais jamais entendus. Ce fut le jour où j'ai perdu confiance
dans le pain de supermarché.

À la maison, j'ai décidé de rechercher la bonne méthode pour faire du pain, et à ma grande surprise,
j'ai appris que les recettes pour faire de la pizza et du pain étaient en fait assez similaires, mais
il y avait aussi des différences. Par exemple, certaines recettes demanderaient de la levure fraîche, alors
que d'autres demanderaient de la levure sèche. En plongeant profondément dans divers forums en ligne et toutes leurs nombreuses
discussions, je suis devenu encore plus confus.
J'ai essayé d'utiliser différentes farines et différentes marques, toutes en versions biologiques et non biologiques.
J'ai alors pris conscience que je ne savais rien sur la fabrication du pain. Les recettes contredisaient souvent chacune,
me laissant encore plus confus. Elles semblaient peu plus qu'une collection de pas apparemment aléatoires
à suivre. Les instructions de cuisson et les températures aussi étaient toutes
différentes\dots
Pendant ce temps, ayant terminé mes études, j'ai commencé à travailler en tant qu'ingénieur.
Nous, les ingénieurs, sommes confrontés à de nombreux défis. Le compilateur ou le runtime
crie toujours des erreurs, et c'est votre travail de comprendre comment les résoudre.
Cela peut prendre des heures, parfois des jours, pour résoudre un simple problème. Si vous voulez
devenir ingénieur logiciel, vous devez développer une certaine attitude de ``ne jamais abandonner''.

Lors de l'écriture de code, les ingénieurs logiciels ont souvent besoin d'utiliser un ensemble de routines préfabriquées. Ces routines ont été
écrites par d'autres ingénieurs et peuvent être utilisées pour livrer du code plus rapidement.
Ce code pré-écrit est communément connu sous le nom de \emph{framework}. Dans de nombreux cas,
ces frameworks ne sont pas construits par une seule personne mais par des ingénieurs du monde entier,
chacun pouvant contribuer en améliorant et en modifiant le code source. Les frameworks ont rendu possible de nombreuses entreprises
prospères.
Dans la plupart des cas, les frameworks font exactement ce qu'ils prétendent faire. Cependant,
parfois vous avez face à des problèmes que vous ne comprenez pas. Dans 99,95\% 
de tous les bugs logiciels, le développeur est le problème. Parfois, cependant, le framework présente un
bug. C'est à ce moment que le développeur doit creuser plus profondément pour voir le \emph{quoi} et le
\emph{pourquoi} de ce que
le framework fait. Vous devrez lire le code source d'autres ingénieurs, et vous serez forcé
de comprendre \emph{pourquoi} les choses se passent.

Mécontent de ce que je faisais cuire, mon esprit d'ingénieur a pris le dessus, et j'ai dû
faire ma propre recherche approfondie pour comprendre ce qui se passait. À ma grande surprise, cependant,
aucune des recettes que j'ai rencontrées ne me dira \emph{pourquoi} je devrais utiliser une quantité X
d'eau et une quantité Y de farine, ou \emph{pourquoi} exactement je devrais utiliser de la levure fraiche plutôt que de la levure sèche. Pourquoi
devrais-je frapper ma pâte en la pétrissant sur le comptoir ? Pourquoi un batteur sur socle
est-il meilleur que le pétrissage à la main ? Pourquoi devrais-je laisser la pâte reposer pendant ce temps?
Pourquoi est-ce important de cuire la pâte à la vapeur ? Ai-je vraiment besoin de
me procurer un four hollandais coûteux pour faire cuire du pain?
Le problème s'est aggravé lorsque j'ai commencé à lire sur le levain. Cela ressemblait à de la magie noire.
Pourquoi certains levains étaient-ils faits à partir de fruits, tandis que d'autres étaient faits à partir de farine?
Pourquoi une recette devrait-elle utiliser du blé alors qu'une autre utilise du seigle ou de l'épeautre ? À quelle fréquence devrait-on
nourrir le levain? Les questions que j'avais alors pourraient prendre 20 pages. J'étais confus,
mais je suis devenu encore plus déterminé à apprendre comment faire du bon pain à la maison.
Les commentaires que j'ai reçus de mes amis m'ont aidé à m'améliorer à chaque
itération de pain fait maison. Comparé à la programmation, où vous devez parfois attendre des mois
pour obtenir des commentaires, la fabrication du pain est beaucoup plus directe. De plus, vous pouvez manger vos succès
(et vos échecs !) Et, à ma grande surprise, même ces échecs étaient meilleurs que
la plupart des pains achetés en magasin. Manger un pain fait maison qui prend des heures à faire vous permet
de développer une relation différente avec votre nourriture, et faire du pain à partir de zéro avec mes
mains nues était un changement bienvenu après des heures de travail sur ordinateur.

J'ai poursuivi mon apprentissage sur le processus de fermentation et les différentes techniques de fabrication du pain.
J'ai abordé le sujet du levain de manière similaire au logiciel, et après des années de
recherche et documentation de mes progrès, j'ai décidé qu'il était temps de partager ces progrès avec le
monde.
Lorsqu'on travaille sur des projets logiciels, il est important de voir leur historique et comment le code source
change avec le temps. De cette façon, vous pouvez facilement revenir aux versions précédentes. C'était
l'outil parfait pour documenter mes recettes, car elles aussi changeraient à chaque
itération suivante. À ma grande surprise, mon travail en open source sur le levain a été apprécié
par d'autres ingénieurs, et le projet est devenu populaire sur le site web GitHub, initialement construit pour
partager des logiciels open source.Maintenant, lorsqu'on cuisine du bon pain, on doit également apprendre certaines techniques. J'ai pensé qu'il serait plus facile de partager ces techniques en vidéo. Ainsi, ma chaîne YouTube est née. J'ai choisi le nom \texttt{The Bread Code} pour capturer mon approche orientée vers l'ingénierie du pain. Cela a pris du temps pour bien faire, mais après avoir choisi des vignettes et des titres plus engageants pour les vidéos que j'ai réalisées, la chaîne a commencé à gagner des spectateurs.
Finalement, trois ans plus tard, je consacre deux jours par semaine à ma passion pour la cuisson du pain, tandis que les trois autres jours, je continue à travailler en tant qu'ingénieur logiciel, écrivant du code au quotidien.

Mes journées de pain me remplissent de joie et de passion. Pour moi, il n'y a rien de mieux que de voir combien de personnes ont réalisé du pain incroyable grâce à mes conseils et explications. La communauté a continué à se développer, générant de nombreuses discussions et idées intéressantes sur le sujet de la fabrication du pain. Il y a toujours quelque chose de nouveau à apprendre, et j'ai le sentiment que même maintenant, je ne grattais que la surface de ce que je sais et enseigne. Auriez-vous jamais imaginé que les mouches des fruits sont comme des abeilles et font partie de l'histoire à succès de la levure sauvage ? J'ai fait une vidéo où j'ai essayé de cultiver des spores de levure sauvage provenant des mouches des fruits afin de faire du pain. Cela a fonctionné ; le pain était incroyablement bon et avait même bon goût ! Ce genre d'expériences pique mon intérêt naturel. Les mener et voir comment d'autres personnes partagent mon intérêt me rend incroyablement heureux.

Le problème avec la gestion d'une chaîne YouTube est que toutes les informations que vous voyez sont filtrées et ensuite fournies à travers un algorithme. Je suis préoccupé par la façon dont les algorithmes façonnent les informations modernes, car ils ont tendance à mettre les utilisateurs dans certaines catégories où ils ne verront ensuite que des actualités liées à ces mêmes catégories fixes. Une métrique clé déterminant la visibilité de votre chaîne est le nombre de personnes qui ont cliqué sur une vidéo après son apparition, et le contenu que vous créez n'est même pas montré à tous les abonnés de votre chaîne. Si l'algorithme détermine que la vidéo n'est pas assez engageante, votre contenu commence à décliner dans le nirvana de YouTube. Même si votre vidéo devient virale, l'algorithme cessera de la montrer dès que les taux d'engagement avec les nouveaux utilisateurs diminuent, et les anciennes vidéos s'estompent au fil du temps à mesure que le facteur de sanction pour déclin augmente. Je le sais, car j'ai développé des algorithmes similaires moi-même en tant qu'ingénieur logiciel.

J'ai depuis décidé de prendre un peu de temps libre du cycle algorithmique pour travailler sur quelque chose de plus long terme et significatif. Ma mission a toujours été de partager mes connaissances avec le plus grand nombre de personnes possible dans le monde. C'est aussi pourquoi mon contenu a été fourni en anglais plutôt qu'en allemand. Après des discussions avec des membres de la communauté, j'ai pensé que l'écriture d'un livre pourrait m'aider à atteindre cet objectif. La plupart des livres qui existent aujourd'hui sont des collections de recettes. Mon idée, cependant, est de vous fournir une base de connaissances plus profonde que vous pouvez utiliser pour suivre d'autres recettes.
En termes de logiciel, ce serait un \emph{cadre de pain}.

Mon objectif pour ce livre est d'aider tous ceux qui rencontrent des problèmes avec la farine, la fermentation, la cuisson, et plus encore. Il devrait fournir une compréhension détaillée de la raison pour laquelle certaines étapes sont nécessaires et comment les adapter lorsque les choses tournent mal lors de la fabrication du pain.
Il est de mon désir que cette connaissance soit accessible à tous dans le monde, quel que soit le budget, et en conséquence, je ne veux pas faire payer pour le livre. C'est pourquoi j'ai décidé de le rendre open source et j'ai demandé à la communauté de soutenir mon travail financièrement via ma page ko-fi \url{https://ko-fi.com/thebreadcode}. Les commentaires de la communauté ont été incroyables jusqu'à présent, et j'ai déjà recueilli beaucoup plus d'argent que je ne l'avais initialement prévu. La version numérique de ce livre restera toujours gratuite. Il existe également une version reliée du livre disponible à l'achat.
Vous pouvez lire plus de détails ici : \url{https://breadco.de/physical-book}Dans ce livre, je vais essayer d'être aussi scientifique que possible. Je ne prétends en aucun cas, cependant, que
ce sera en lui-même un travail scientifique. J'ai mené plusieurs expériences dont je vais parler
ici, mais pour vraiment appeler cela de la science, vous auriez probablement besoin de répéter la même expérience
un millier de fois dans un environnement de laboratoire, ce que je n'ai pas fait. Je vais faire de mon mieux, cependant, pour fournir
des références scientifiques là où c'est possible et pour distinguer clairement entre les faits et l'opinion personnelle.

J'espère que vous vous amuserez à lire ceci et que vous en apprendrez davantage sur le monde fascinant de la fabrication du pain
et c'est mon souhait sincère que ce travail vous fournisse la chaîne d'outils solide que j'aurais souhaité
avoir accès lorsque j'ai commencé mon propre voyage avec le pain.

Merci.

Hendrik